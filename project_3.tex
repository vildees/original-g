\documentclass[a4paper,12pt, english]{article}
\usepackage[T1]{fontenc}
\usepackage[utf8]{inputenc}
\usepackage{graphicx}
\usepackage{babel}
\usepackage{amsmath}
\usepackage{ulem}
\usepackage{a4wide}
\usepackage{graphicx}
\usepackage{listings}
\usepackage{tabularx}
\usepackage{tabulary}

\begin{document}

\begin{titlepage}
\begin{center}
\textsc{\Large Computational Physics, Project 3}\\[0.5cm]
\textsc{Vilde Eide Skingen and Kari Eriksen}\\[0.5cm]

\end{center}
\end{titlepage}

\begin{abstract}
The aim of project 3 is to create a code for simulating the solar system. In the first part we will look at a hypothetical solar system consisting of the Sun and the Earth. We will assume that the Sun's mass is sufficiently large, so that its motion can be neglected. With this assumption we will compute the motion of the Earth using different methods for solving ordinary differential equations.  
\end{abstract}

\section{A model for the solar system}

\subsection{Introduction}
Our goal is to calculate the position of Earth as a function of time.
\subsection{Theory}
In the first part of the project we will look at a hypothetical solar system with only one planet, the Earth, in orbit around the sun. The only force in this problem is gravity, given by Newton's law of gravitation $$ F_G = \frac{GM_{sun}M_{Earth}}{r^2} $$
where $M_{sun}$ and $M_{Earth}$ are the masses of the Sun and Earth, $r$ the distance between them, and $G$ the gravitational constant. If we assume that the Sun has a much larger mass than the mass of the Earth, we can safely neglect the motion of the sun in the calculations. 

We will also assume that the orbit of Earth around the Sun is co-planar, and that it lies in the $xy$-plane. We  then get the following equations from Newton's second law of motion
$$\frac{d^2x}{dt^2} = \frac{F_{G,x}}{M_{Earth}}$$
$$\frac{d^2y}{dt^2} = \frac{F_{G,y}}{M_{Earth}}$$
where $F_{G,x}$ and $F_{G,y}$ are the $x$ and $y$ component of the gravitational force. 

From figure .... we have that

$$F_{G,x} = - F_G  \hspace{0.7 mm} cos \theta = - \frac{GM_{sun}M_{Earth}}{r^2} \hspace{0.7 mm} cos \theta $$
$$F_{G,y} = - F_G \hspace{0.7 mm} sin \theta = - \frac{GM_{sun}M_{Earth}}{r^2} \hspace{0.7 mm} sin \theta $$

and figure ....

$$ cos \theta = \frac{x_i}{r} $$
$$ sin \theta = \frac{y_i}{r} $$

We can rewrite the second-order ordinary differential equations as a set of coupled first order differential equations. 
We know that the velocity is connected to the position by the time derivative so that
$$ \frac{dx}{dt} = v_x \hspace{20 mm} \frac{dy}{dt} = v_y$$
and thus we have that
$$ \frac{dv_x}{dt} = \frac{F_{G,x}}{M_{Earth}} \hspace{20 mm} \frac{dv_y}{dt} = \frac{F_{G,y}}{M_{Earth}} $$

These ordinary differential equations can be solved by numerical methods. However, before doing so, we should find a suitable choice of units. We choose to use the astronomical units in our project. The astronomical unit of length is given to be the average distance between the Sun and the Earth, $1 AU = 1.5*10^{11} m$. It is convenient to measure time in years, since this better suits the cycles of the solar system. 
We also need the corresponding unit of mass. To a very good approximation, the Earth's orbit around the Sun is circular. For circular motion we know that the force must obey the following relation 
$$\frac{M_{Earth}v^2}{r} = F_G = \frac{GM_{sun}M_{Earth}}{r^2}$$ where $v$ is the velocity of the Earth. 
Since the orbit is circular the velocity of the Earth is $v = \frac{2 \pi r}{1 year}$. Since $r$ is the distance between the Sun and Earth we know that $r = 1 AU$. Hence $v = 2 \pi (AU/years)$. Rearranging the      equation we get that
$$v^2r = GM_{sun} = 4 \pi ^2 AU^3/years^2$$

In preparation to constructing a computational solution, we convert the equations of motion into a discretized set of equations. We find

\begin{center}
\begin{equation}
v_{x,i+1} = v_{x,i} - \frac{2 \pi ^2 x_i}{r_i ^3} \Delta t 
x_{i+1} = x_i + v_{x,i+1} \Delta t 
v_{y,i+1} = v_{y,i} - \frac{4 \pi ^2 y_i}{r_i ^3} \Delta t 
y_{i+1} = y_i + v_{y,i+1} \Delta t 
\end{equation}
\end{center}

where $\Delta t$ is the time step and $4 \pi ^2 = GM_{sun}$.


\end{document}    